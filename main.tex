\documentclass[letterpaper, 10 pt, conference]{ieeeconf}  % Comment this line out
                                                          % if you need a4paper
%\documentclass[a4paper, 10pt, conference]{ieeeconf}      % Use this line for a4
                                                          % paper

\overrideIEEEmargins
% See the \addtolength command later in the file to balance the column lengths
% on the last page of the document

\usepackage[utf8]{inputenc}

\title{Centralized Identity Authority Through Blockchain}
\author{William Shupe, Evan Hirn, Caleb Edwards}
\date{April 2020}

\begin{document}

\maketitle

\section{Introduction}
% Motivate  and  introduce  the  problem  you  are  solving. Very briefly summarize your methodology, your experiments, and your important results.

Currently there are no wide scale, secure identity platforms where an individual can prove who they are in a timely and secure fashion. Right now, the United States uses things like social security numbers, but these numbers can be stolen easily. We would like to provide a solution where a person can prove their identity, citizenship, and potentially
other things like credit without having to worry about losing permanent identifiers like a social security number.

It takes a lot of work to reconcile the problem of losing one of these permanent identifiers. When a person in the United States loses one of these identifiers that represent their identity, it could take weeks for that central authority to fix the problem and provide new identification. This is a problem that needs fixed, and blockchain is the answer.

The implementation that was chosen, was being able to add new user data to the chain, edit user data, and share user data. This implementation was chosen because it can show without too much complexity that identity resolution through blockchain is a viable way of proving a user's identity. 

\section{Related Work}
% Related work –Summarize the existing work related to your problem.

There are many different identity solutions that have been created through blockchain \cite{1}.

\section{Adversary Model}
% Adversary model –Describe the adversary you are securing against.

\section{Methodology}
% Methodology –Describe your solution and any other methods you have used to solve your problem.

It was decided to implement a central authority in blockchain. The nodes will be controlled by the central authority. The nodes have the ability to add, edit, and share information with certified service providers. 

\section{Implementations/Experimentation}
% Implementations/Experimentation –Describe  your  experiments  and  your results. Discuss your results.

\section{Conclusion}
% Conclusion –Summarize  your  work  and  your  results.  Indicated  any  future directions.

\bibliographystyle{IEEEtran}
\bibliography{bibliography}
% References –List all the references you have used for this work.

\end{document}
